%! Author = magnus.silverdal
%! Date = 2020-11-23

% Preamble
\documentclass[11pt]{beamer}
\usetheme{Copenhagen}
% Packages
\usepackage{amsmath}

\title{Repetition Kapitel 2 Harmonisk svängningsrörelse}
\author{Magnus Silverdal}
\institute{NTI Gymnasiet}
\date{\today}
% Document
\begin{document}
    \frame{\titlepage}

    \begin{frame}
        \frametitle{Harmonisk svängning}

        \begin{block}{Villkor för harmonisk oscillator}
            Om kraften är proportionell mot avvikelsen från ett jämviktsäge finns förutsättingar för en harmonisk
            svängning, en oscillator. Matematisk betyder det att kraften är en linjär funtion
            \begin{equation}
                F = k \cdot x
            \end{equation}
            Newtons andra lag $ F = m \cdot a$ ger, när de två uttrycken sätts lika med varandra
            \begin{equation}
                F = k \cdot x = m \cdot a = F
            \end{equation}
            Eftersom $a = \frac{d^2 x}{dt^2} = x''$ så blir resultatet en differentialekvation
            \begin{eqnarray}
                k \cdot x &=& m \cdot x'' \text{eller} \\
                m \cdot x'' - k \cdot x &=& 0 \label{harmosc}
            \end{eqnarray}
        \end{block}
    \end{frame}
    \begin{frame}
        \begin{block}{}
            Lösningen till denna ekvation är en trigonometrisk funktion
            \begin{equation}
                x(t) = A \sin{\omega t} \label{sol}
            \end{equation}
            Beroende på kraften så får lösningen olika uttryck för $\omega$. För en fjäder i en vikt är kraften, enligt
            Hooks lag, $f = k x$ där $k$ är fjäderkonstanten. Insättning av lösningen (\ref{sol}) i (\ref{harmosc}) ger uttrycket
            \begin{equation}
                \omega = \sqrt{\frac{k}{m}}
            \end{equation}
            För en matematisk pendel kan kraften längs banan (cirkelbågen) istället skrivas $- m g \sin{\frac{x}{l}}$. För små vinklar
            gäller att $\sin{x} \approx x$. Det ger samma lösning (\ref{sol}) men med en annan vinkelhastighet
            \begin{equation}
                \omega = \sqrt{\frac{g}{l}}
            \end{equation}
        \end{block}
    \end{frame}
    \begin{frame}
        \begin{block}{Vinkelhastighet}
            I lösningen till den harmoniska oscillatorn (\ref{harmosc}) är vinkelhastigheten $\omega$ central. I praktiken
            är den svår att observera. Vanligare är att mäta periodtiden $T$ eller frekvensen $f$.
            \begin{equation}
                \omega = 2 \pi f = \frac{2 \pi}{T}
            \end{equation}
        \end{block}
    \end{frame}
    \begin{frame}
        \begin{block}{Fortskridande vågor}
            Ekvationen (\ref{sol}) beskriver en våg som utbreder sig. När en våg rör sig ges hastigheten av $v = f \lambda$.
            Detta är våghastigheten. För ljud är det ljudhastigheten, för ljus ljushastigheten och för en våg på en sträng
            ges den av $v = \frac{F}{\rho A}$. Den är i princip oberoende av vågen.

            Vågor som rör sig kan reflekteras och transmitteras när de rör sig mellan olika medier, dvs om våghastigheten skulle förändras.
            Vågor som träffar andra vågor genomgår superposition och om de är i fas kan resonsns eller stående vågor uppstå.
            En stående våg inträffar när refekterade vågor hamna i fas. För att det ska hända måste våglängden stämma med mediets längd.
            En stående våg består av noder och bukar. En nod är den del av vågen som inte rör sig, buken är den del som svänger mest.
            Beroende på om mediet är öppet eller slutet i ändarna blir villkoret för stående våg olika.
            \end{block}
        \end{frame}
    \begin{frame}
        \begin{block}{}
            För en helt öppen eller helt sluten situation gäller villkoret för längden $l$ och våglängden $\lambda$
            \begin{equation}
                l = n \frac{\lambda}{2} , n = 1,2,3...
            \end{equation}
            och för en halvöppen situation
            \begin{equation}
                l = \frac{(2n-1)\lambda}{4}  , n = 1,2,3...
            \end{equation}
        \end{block}
    \end{frame}
    \begin{frame}
        \begin{block}{Ljud}
            Ljudets intensitet $I$ är ett mått på hur mycket energi som når lyssnaren
            \begin{equation}
                I = \frac{P}{A}
            \end{equation}
            Ljudnivån $L$ bestäms som det logaritmiska förhållandet mellan ljudinstensiteten och den lägsta hörbara intensiteten $I_0 = 10^{-12}$.
            \begin{equation}
                L = 10 \log{\frac{I}{I_0}}
                \end{equation}
        \end{block}
    \end{frame}
\end{document}