%! Author = magnus.silverdal
%! Date = 2021-08-26

% Preamble
\documentclass[11pt]{article}

% Packages
\usepackage{amsmath}
\usepackage{hyperref}

% Document
\begin{document}
\section{Vad}
    Den här instruktionen hjälper dig att komma igång med projektorerna i ALC, framförallt om du har lektioner där
    och vill utnyttja alla projektorer samtidigt.
\section{Hur}
    Du behöver installera programvaran EasyMP Network Projection från Epson som du hittar på den \href{https://ftp.epson.com/drivers/epson16189.exe}{här länken}.
    Efter installationen startar du programmet (vanligtvis hamnar det i en mapp som heter Epson under startmenyn). Välj avancerat anslutningsläge.
    Efter en stund kommer programmet meddela att att det inte hittar några projektorer. Välj avbryt direkt eller vänta en stund och tryck fortsätt.

    Första gången du kör programmet behöver du leta reda på projektorerna på nätverket. Välj manuell sökning och skriv in projektorns IP-nummer.
    Projektorerna i ALC ligger på IP-nummer 10.80.44.33 till 10.80.44.36. När den första hittats gör du en ny sökning på nästa tills du har alla fyra i listan.
    Tänk på att projektorn måste vara påslagen för att den ska kunna hittas. Fjärrkontrollen funkar till alla fyra så du får sikta
    bra om du bara vill slå av eller på en projektor, whiteboarden funkar som en jättebra spegel för signalen från fjärrkontrollen.

    När du har alla projektorer i listan kan du spara profilen så att du slipper söka efter dem varje gång. Istället kan du då välja en profil i listan.

    Kryssa nu i de projketorer du vill ansluta till och valj anslut. Efter en liten stund dyker det upp ett litet verktygsfält
    och du delar din skärm med projektorn. För att det ska synas måste projektorn vara i läge LAN, det finns en knapp på fjärrkontrollen för att ställa in detta.

    Nu borde allt fungera.
\section{Och sen då}
    I den lilla kontrollpanelen kan du styra projektorerna. Den viktigaste funktionen är pause som fryser bilden på projektorn och låter dig
    göra annat på din skärm som att läsa mailen. Det går inte att köra i utvidgat läge utan det blir alltid din egen skärm som skickas till projektorn.
    När du är klar kopplar du ifrån och stänger av projektorerna.
\section{Var det allt?}
    Det finns andra verktyg till projektorerna.
    \subsection{Easy MP Multi PC Projection }
        Multi PC Projection ger dig möjlighet att administrera elevernas koppling till projektorn.
        Du kan välja vilken elev som ska visas på vilken projektor och dela upp projektorn så att upp till fyra elever kan visa sina skärmar samtidigt.
    \subsection{Easy MP Interactive Tools}
        Interactive Tools kopplar ihop den digitala whiteboarden med datorn och gör så att det som skrivs och ritas kan sparas.
        Bra både för genomgångar och för elevsamarbeten. Programmet ger också möjlighet att använda pennorna och whiteboarden som en
        en jättepekskärm.
\end{document}