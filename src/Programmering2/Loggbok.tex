%! Author = magnus.silverdal
%! Date = 2020-11-30

% Preamble
\documentclass[11pt]{article}

% Packages
\usepackage{amsmath}
\usepackage{graphicx}
\begin{document}
    \begin{titlepage}
        \centering
        \includegraphics[width=0.15\textwidth]{../images/logo.png}\par\vspace{1cm}
        {\scshape\LARGE Programmering 2 \par}
        \vspace{1cm}
        {\scshape\Large Projekt 2\par}
        \vspace{1.5cm}
        {\huge\bfseries Loggbok\par}
        \vspace{2cm}
        {\Large\itshape Ett objektorienterat system med koppling till filer och databaser\par}
        \vfill
        Utformad av\par
        Magnus \textsc{Silverdal}

        \vfill

% Bottom of the page
        {\large \today\par}
    \end{titlepage}
    \section{Inledning}
        Det har nu blivit dags att sätta samman all er kunskap till ett fungerande projekt. Tanken är att ni ska börja
        från början och använda era lärdommar för att skapa en bra loggboksapplikation. Ni har redan, i  olika utsträckning,
        löst delar av uppgiften men i efterhand står det ofta klart att programmet hade kunna skrivas bättre. Speciellt
        eftersom ni inte hade hela problemet klart för er när ni startade.

    \section{Uppgiften}
        Er uppgift är att skriva ett användbart loggbokssytem. Loggbokens centrala delar är att kunna skriva noteringar som taggas med
        författare och datum. Systemet ska vara stabilt och tillförlitligt.
        \subsection{Redovisning}
            Projektet redovisas som ett projekt i github. Era commits och projektbrädet kommer att vara centrala för att
            redovisa processen. De tekniska delarna redovisas i den dokumentation och utvärdering som avslutar projektet.
        \subsection{Obligatoriska delar}
            \begin{enumeration}
                \item [OOP] Loggboken, datat i den och funktionerna ska lösas med objektorientering
                \item [MVC] Ni ska arbeta för att ert program ska vara uppbyggt enligt Model-view-controller-tanken.
                Det betyder att det grafiska gränssnittet ska vara oberoende av loggbokens inre struktur.
                \item [Databas] Loggboken ska också kopplas till en databas. Alla inlägg som görs ska sparas i databasen
                och programmet ska läsa in den information som ligger i databasen när det startas
                \item [Filhantering] Loggboken ska gå att spara till fil och återställ från fil. Observera att detta kan
                kollidera med databaskopplingen.

            \end{enumeration}
        \subsection{Fördjupningar och extra uppgifter}
            \begin{enumeration}
                \item Användarvänligt GUI
                \item Commandline-läge
                \item Arbeta med olika loggböcker. Spara till olika filer.
                \item Tillåta att flera användare skriver i samma loggbok \"samtidigt\".
                \item Användare med inloggning (databas)
                \item Skapa nya loggböcker på databas-servern
                \item Skapa en generell lösning för databas-kopplingen
            \end{enumeration}
    \section{Tidsplan}
        Ni kommer att arbeta med utvecklingen av loggboken fram till jul. Det betyder 6st 80-minuterspass. Ni får planera
        lägga upp ert arbete så att ni blir klara i tid. Använd ett projektbräde i github för att hålla koll. Fila inte
        allt för mycket på detaljer utan se till att ni uppfyller kraven först.

        Efter jul kommer ni att skriva en Post Mortem och arbeta med dokumentation
    \section{Bedömning}
\end{document}