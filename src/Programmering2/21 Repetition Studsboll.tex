%! Author = magnus.silverdal
%! Date = 2021-08-26

% Preamble
\documentclass[11pt]{article}

% Packages
\usepackage{amsmath}

% Document
\begin{document}
\section{Syfte}
    Syftet med den här uppgiften är att repetera några av de viktigaste delarna av programmering 1 som ni kommer att behöva
    i programmering 2. Beroende på hur mycket ni kan finns flera steg som ni kan gå vidare med men det viktiga är att ni har koll på grunderna.
\section{Bakgrund}
    Det absolut viktigaste är att ni har koll på
    \begin{itemize}
        \item Variabler
        \item If-satser (selektion)
        \item Loopar (iteration)
        \item Metoder
        \item Hur program skrivs, körs och felsöks i IntelliJ och CMD
    \end{itemize}
    Om du känner att du behöver mer material för att läsa in dig på detta rekommenderar jag antingen
    \href{https://classroom.udacity.com/courses/ud282}{Den här kursen på Udacity} som går igenom precis dessa delar eller
    de första kapitlen i vår referenslitteratur till kursen som du hittar i classroom.
\section{Uppgiften}
    Din uppgift är att skriva ett simuleringsprogram för en studsande boll. I grundläget gäller kraven i steg 1, du kan sedan
    välja att göra en eller flera av de andra stegen i valfri ordning.
    \subsection{Steg 1}
        Ditt program ska generera en värdetabell med tid och höjd för en studsande boll. Du ska lösa det genom att använda ett
        fast tidssteg, en acceleration, en hastighet och en höjd som uppdateras i steg. Bollen studsar när den når golvet.
        Den enklaste lösningen är att låta hastigheten byta riktning när höjden blir mindre än eller lika med 0. Ditt
        program måste innehålla tre metoder, en för uppdatera hastigheten, en för att uppdatera höjden och en för att kolla studsen.
        Tänk på vilka variaber som behövs i programmet och vilka som ska skicka in i de olika metoderna. Tänk också på hur
        du ska få tillbaka de nya värdena! Du ska inte använda någon klass för bollen, det ska vi titta på senare. Stoppa in
        resultatet av ditt program (värdetabellen med tid och höjd) i excel och rita grafen ($h(t)$).
    \subsection{steg 2}
        Lägg till en studskoefficient till studsen så att inte hastigheten bara byter tecken!
    \subsection{steg 3}
        Låt bollen röra sig i flera riktningar. I sidled har du konstant hastighet så du behöver ingen acceleration eller
        uppdateringsmetod för sidohastigheten. Nu får du rita grafen till y(t) och x(t) som en kurva. Glöm inte studsen mot väggarna...
    \subsection{steg 4}
        Ta hand om problemet att bollen hamnar under golvet. Om höjden blir negativ måste du räkna baklänges tills du hittar
        tiden när den slår i golvet och räkna ut den nya hastigheten där.
    \subsection{steg 5}
        Vad händer om golvet lutar? Studsen sker alltid mot normalen av golvet så det går att ta hänsyn till golvets
        lutning när bollen studsar. Nu börjar det bli knepigt...
    \subsection{steg 6}
        Använd animeringsmotorn från förra året för att visa hur simuleringen ser ut.
\section{Redovisning}
    Lämna in länken till repot i uppgiften. Jag kommer att ge feedback i github allt eftersom.

\end{document}