%! Author = magnus.silverdal
%! Date = 2020-12-03

% Preamble
\documentclass[11pt]{beamer}
\usetheme{Copenhagen}
% Packages
\usepackage{amsmath}

\title{Repetition Kapitel 5}
\author{Magnus Silverdal}
\institute{NTI Gymnasiet}
\date{\today}
% Document
\begin{document}
    \frame{\titlepage}

    \begin{frame}
        \frametitle{Arbete}
        Inom mekanik (läran om saker som rör sig) behövs en kraft $F$ för att förändra hur föremålen rör sig, enligt Newtons andra lag.
        Om det sker en förändring har ett arbete $W$ utförts och det definieras som
        \begin{equation}
            W = F \cdot s
        \end{equation}
        där $s$ är sträckan som kraften verkat längs. Arbetets enhet blir $Nm$

        \vspace{10}
        Detta gäller endast om kraften är konstant, det generella uttrycket är
        \begin{equation}
            W = \int F(s) ds
        \end{equation}

    \end{frame}

    \begin{frame}
        \frametitle{Energi}
        Arbete är en förändring av energin hos ett föremål. Inom Mekaniken betyder det att när en kraft förändrar
        föremålets rörelse så förändras föremålets energi. Eftersom energi inte kan skapas eller förstöras är arbete
        ett sätt att flytta energi från en form eller plats till en annan
        \begin{equation}
            W = \Delta E
        \end{equation}
        Detta gäller för det eller de föremål som studeras. Enheten för energi är Joule och betecknas $J$ men enligt
        formeln syns att det är exakt samma sak som $Nm$
    \end{frame}

    \begin{frame}
        \frametitle{Mekanisk energi}
        De energier som är relevanta i ett mekanisk system är rörelseenergi, eller kinetisk energi
        \begin{equation}
            E_k = \frac{mv^2}{2}
        \end{equation}
        och lägesenergi eller potentiell energi
        \begin{equation}
            E_p = mgh
        \end{equation}
        All energi är relativ någon 0-nivå så när vi jämför energier måste vi tänka på att de ska ha samma nollnivå.
        I praktiken innebär det att höjd och hastighet ska mätas på samma relativa sätt.
    \end{frame}
    \begin{frame}
        \frametitle{Effekt och verkningsgrad}
        För att kunna jämföra och räkna på olika arbeten räcker det inte alltid att veta slutresultatet, det kan också
        spela roll hur lång tid arbetet tagit. Ett mått på detta är effekt. Definitionen på effekt är energiändring per tidsenhet
        \begin{equation}
            P = \frac{\Delta E}{\Delta t}
        \end{equation}
        För att beskriva hur stor del av den energi som behövs för ett arbete jämfört med hur mycket arbete som utförs
        används begreppet verkningsgrad.  Det är kvoten mellan de nyttiga, utförda arbetet och den tillförda energin.
        Eftersom effekten är arbete per tidsenehet och arbetet utförs under en given tid kan detta också skrivas som
        \begin{equation}
            \eta = \frac{E_{nyttig}}{E_{tillförd}} = \frac{P_{nyttig}}{P_{tillförd}}
        \end{equation}

    \end{frame}
\end{document}