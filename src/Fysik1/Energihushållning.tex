%! Author = magnus.silverdal
%! Date = 2021-02-18

% Preamble
\documentclass[11pt]{article}

% Packages
\usepackage{amsmath}
\usepackage{fancyhdr}
\usepackage{hyperref}

% Data
\title{Energihushållning \protect\\ uppgift i vetenskapligt skrivande\protect\\Fysik 1}

\author{Magnus Silverdal}
\def\inst{Teknikprogrammet}
\def\typeofdoc{Vetenskaplig rapport}
\def\course{Fysik 1}
\def\name{Magnus Silverdal}
\def\username{Magnus.Silverdal}
\def\email{\username{}@ga.ntig.se}
\def\graders{Magnus Silverdal}

% Header and footer
\lfoot{\footnotesize{\name, \\ \email}}
\rfoot{\footnotesize{\today}}
\lhead{\sc\footnotesize{Energihushållning}}
\rhead{\nouppercase{\sc\footnotesize\rightmark}}
\pagestyle{fancy}
\renewcommand{\headrulewidth}{0.2pt}
\renewcommand{\footrulewidth}{0.2pt}

% Document
\begin{document}
    \maketitle
    \section{Uppgift}
    Ni ska skriva en vetenskaplig rapport inom området energiförsörjning. I rapporten ska ni beskriva
    olika sätt att utvinna energi ur bränslena, påverkan på miljön lokalt och globalt samt hur de
    påverkar samhället lokalt och globalt.
    Den tekniska beskrivningen av hur energiutvinningen fungerar ska vara baserad på vetenskap,
    använd era fysikkunskaper.
    Miljöpåverkan bör ta upp både den aktuella frågan om global uppvärmning men också miljögifter
    och markförstöring.
    Påverkan på samhället bör diskutera ekonomiska faktorer
    samt kopplingen till konflikter om resurser i världen.
    \section{Genomförande}
    Till att börja med ska ni välja vilken energiform ni ska undersöka. Ni kan välja mellan följande fyra
    \begin{enumerate}
        \item Fossila bränslen (Kol, olja och naturgas)
        \item Förnybara bränslen, biobränslen (Skog, Sopor mm)
        \item Kärnbränslen (Uran)
        \item Naturliga energikällor (Sol, vind och vatten)
    \end{enumerate}
    Börja med att formulera en frågeställning så att du vet vad du behöver för material, leta sedan upp
    de källor som behövs. Att bara använda Wikipedia är inte nog för en vetenskaplig rapport, du måste
    leta djupare. Var noga med källkritiken! Fårgeställningen är din specifika inriktning i uppgiften.
    Välj någon teknik och några aspekter för påverkan på miljö och samhälle.

    \subsection{Material}
    Här följer några källor som kan vara användbara för att hitta information
    \begin{itemize}
        \item Kursbokens kapitel 8
        \item \href{https://www.diva-portal.org}{DiVA}, här samlas alla vetenskapliga publikationer som ges ut av svenska universitet och högskolor.
        \item \href{https://www.umu.se/bibliotek/}{Universitetsbibliteket i Umeå} har många artiklar och e-böcker. Om ni hittar något som inte är open access kan ni säga till så hämtar jag det åt er.
        \item \href{https://www.minabibliotek.se}{Stadsbiblioteket} har både fysiska böcker att låna och e-böcker att läsa online.
        \item \href{https://scholar.google.com/}{Google Scholar}, här finns mycket, men det är svårt att hitta guldkornen.
    \end{itemize}

    Det är viktigt att er rapport
    \begin{itemize}
    \item följer de mallar som finns på NTI, att språket är vetenskapligt (samma som i en labrapport),
    \item att det finns en bra källhantering med referatmarkörer och källförteckning
    \item och att rapporten har en bra struktur (framsida,(sammanfattning),innehållsförteckning,
    frågeställning/syfte, rapport, diskussion, källförteckning)
    \end{itemize}

    Bra stöd för vetenskapligt skrivande finns på UBs hemsida: \href{https://www.umu.se/bibliotek/soka-skriva-studera/akademiskt-skrivande/}{akademisk skrivande},
    \href{https://www.umu.se/bibliotek/soka-skriva-studera/skriva-referenser/}{att skriva referenser}
    \section{Inlämning}
    Rapporten lämnas in på classroom i form av en pdf-fil.
    \pagebreak
    \subsection{Koppling till kursplanen}
    De förmågor som bedöms under projektet är
    \begin{itemize}
        \item Kunskaper om fysikens begrepp, modeller, teorier och arbetsmetoder samt förståelse av hur
    dessa utvecklas.
        \item Kunskaper om fysikens betydelse för individ och samhälle.
        \item Förmåga att använda kunskaper i fysik för att kommunicera samt för att granska och
    använda information.
   \end{itemize}
    De delar av det centrala innehållet som behandlas är
    \begin{itemize}
    \item Energi och energiresurser
    \item Arbete,effekt, potentiell energi och rörelseenergi för att beskriva olika energiformer:
    mekanisk, termisk, elektrisk och kemisk energi samt strålnings- och kärnenergi.
    \item Energiprincipen, entropi och verkningsgrad för att beskriva energiomvandling,
    energikvalitet och energilagring.
    \item Termisk energi: inre energi, värmekapacitet, värmetransport, temperatur och
    fasomvandlingar.
    \item Kärnenergi: atomkärnans struktur och bindningsenergi, den starka kraften, massa-
    energiekvivalensen, kärnreaktioner, fission och fusion. Energiresurser och energianvändning
    för ett hållbart samhälle.
    \item Fysikens karaktär, arbetssätt och matematiska metoder.
    \item Vad som kännetecknar en naturvetenskaplig frågeställning.
    \item Hur modeller och teorier utgör förenklingar av verkligheten och kan förändras över tid.
    \item Utvärdering av resultat och slutsatser genom analys av metodval, arbetsprocess och
    felkällor.
    \item Ställningstaganden i samhällsfrågor utifrån fysikaliska förklaringsmodeller, till
    exempel frågor om hållbar utveckling.
    \end{itemize}

\end{document}