%! Author = magnus.silverdal
%! Date = 2021-03-25

% Preamble
\documentclass[11pt]{article}

\usepackage{amsmath}
\usepackage{graphicx}
\usepackage{fancyhdr}

%data
\title{Laboration Ellära: Ohms lag}
\author{Magnus Silverdal}
\def\inst{Teknikprogrammet}
\def\typeofdoc{}
\def\course{Fysik 1}
\def\name{Magnus Silverdal}
\def\username{Magnus.Silverdal}
\def\email{\username{}@ga.ntig.se}
\def\graders{Magnus Silverdal}

%Sidhuvud och sidfot
\lfoot{\footnotesize{\name, \\ \email}}
\rfoot{\footnotesize{\today}}
\lhead{\sc\footnotesize\title}
\rhead{\nouppercase{\sc\footnotesize\leftmark}}
\pagestyle{fancy}
\renewcommand{\headrulewidth}{0.2pt}
\renewcommand{\footrulewidth}{0.2pt}

% Document
\begin{document}
    \maketitle
    \section{Del 1: Ohms lag}
    \subsection{Syfte}
    Syftet med den här delen av laborationen är att experimentellt ta fram sambandet mellan ström och spännning.
    Ditt mål är att hitta strömmen $I$ som en funktion av spänningen $U$, $I = f(U)$.
    \subsection{Metod och material}
    Du behöver
    \begin{enumeration}
        \item Spänningskälla (kub)
        \item kablar
        \item motstånd monterat i fäste
        \item Voltmeter (Pasco)
        \item Amperemeter (Pasco)
        \item Dator
        \item SparkVUE programvara (finns på classroom)
    \end{enumeration}
    \subsubsection{Utförande}
    Börja med att sätta fast motståndet i krokodilklämmorna. Koppla sedan inhop en krets med
    kuben-motståndet-ampermetern-kuben i serie (efter varandra). Koppla sedan in voltmetern parallellt med motståndet (över)
    Ta upp en mätserie där du varierar spänningen från 0 till 10 volt. Analysera resultatet (regressionsanalys) och ta fram ett uttryck för
    $I = f(U)$. Analysera de konstanter du får (enhet och fysikalisk betydelse).

    \section{Del 2: resistansen hos en tråd}
    Syftet med den här delen av laborationen är att bekräfta sambandet mellan resistansen i en tråd och de variabler som
    ingår i uttrycket $R = \rho \frac{l}{A}$.
    Ditt mål är att utföra mätningar som bekräftar formeln.
    \subsection{Metod och material}
    Du behöver
    \begin{enumeration}
        \item Spänningskälla (kub)
        \item kablar
        \item olika trådar (nickel och stål)
        \item 2 multimetrar
        \item Dator
    \end{enumeration}
    \subsubsection{Utförande}
    Börja med att sätta fast en av trådarna i krokodilklämmorna. Koppla sedan inhop en krets med
    kuben-motståndet-ampermetern-kuben i serie (efter varandra). Koppla sedan in voltmetern parallellt med tråden (över)
    Mät ström och spänning. Variera längden av tråden och upprepa mätningen. Variera tjockleken av tråden (hur?) upprepa mätningen.
    Byt till en anna tråd, upprepa mätningen. Jämför dina mätningar med formlen. Stämmer den?
\end{document}